\documentclass[12pt, a4paper]{article}
\usepackage[utf8]{inputenc}
\usepackage[T1]{fontenc}
\usepackage[brazil]{babel}
\usepackage{geometry}
\geometry{a4paper, left=3cm, right=2cm, top=3cm, bottom=2cm}
\usepackage{setspace}
\onehalfspacing
\usepackage{indentfirst}
\usepackage{ragged2e}
\usepackage{enumitem}

\begin{document}

\begin{flushleft}
\textbf{Nome:} Giovanni Müller \\
\textbf{RA:} 1002351335 \\
\textbf{Tema:} Economia colaborativa na produção de pães em Palmeira-PR \\
\textbf{Data:} \today \\
\end{flushleft}

\section*{\centering Fundamentação Teórica: Economia Colaborativa na Produção de Pães em Palmeira-PR}

\justifying

\section{Introdução}

A economia colaborativa representa um modelo econômico baseado no compartilhamento, troca e colaboração entre indivíduos e comunidades, configurando-se como uma alternativa aos modelos tradicionais de produção e consumo. Este trabalho tem como objetivo identificar autores fundamentais que tratam desta temática, com foco específico na aplicação na produção de pães no município de Palmeira-PR.

\section{Autor Essencial e Obra Referencial}

\subsection*{Autor Escolhido}
Rachel Botsman

\subsection*{Obra Referencial}
\textbf{Título:} What's Mine is Yours: The Rise of Collaborative Consumption \\
\textbf{Ano:} 2010 \\
\textbf{Editora:} HarperBusiness \\
\textbf{Cidade:} New York

\section{Justificativa da Escolha}

A escolha de Rachel Botsman como autora essencial para o tema "Economia colaborativa na produção de pães em Palmeira-PR" fundamenta-se nos seguintes aspectos:

\begin{itemize}[leftmargin=*, nosep]
    \item \textbf{Pioneirismo Conceptual:} Botsman é amplamente reconhecida como uma das principais teóricas e pioneiras no estudo da economia colaborativa, tendo cunhado o termo "consumo colaborativo";
    
    \item \textbf{Aplicabilidade Prática:} Sua obra oferece framework teórico aplicável à realidade de pequenas comunidades, como o caso de Palmeira-PR;
    
    \item \textbf{Abordagem Sistemática:} A autora apresenta modelos de negócios colaborativos que podem ser adaptados à produção artesanal de alimentos;
    
    \item \textbf{Relevância Local:} Os conceitos de "reputação coletiva" e "confiança compartilhada" desenvolvidos por Botsman são particularmente relevantes para comunidades menores;
    
    \item \textbf{Sustentabilidade:} A obra aborda a dimensão da sustentabilidade econômica e ambiental, crucial para negócios alimentícios locais;
    
    \item \textbf{Casos Práticos:} O livro apresenta estudos de caso de iniciativas colaborativas em setores alimentícios que servem como referência para a realidade paranaense.
\end{itemize}

\section{Relação com o Tema Específico}

A obra de Botsman fornece a base teórica para compreender como princípios da economia colaborativa podem ser aplicados na produção de pães em Palmeira-PR através de:

\begin{itemize}[leftmargin=*, nosep]
    \item Modelos de compartilhamento de fornos comunitários;
    \item Sistemas de troca de ingredientes entre produtores locais;
    \item Redes colaborativas de distribuição e comercialização;
    \item Compartilhamento de conhecimentos e técnicas tradicionais;
    \item Estruturas de cooperativismo moderno adaptado à produção artesanal.
\end{itemize}

\section{Considerações Finais}

Rachel Botsman oferece o instrumental teórico fundamental para analisar e propor modelos de economia colaborativa aplicáveis à produção de pães em Palmeira-PR. Sua obra fornece tanto a base conceptual quanto exemplos práticos que permitem adaptar os princípios do consumo colaborativo às especificidades da realidade local, tornando-se assim referência essencial para esta pesquisa.

\end{document}
