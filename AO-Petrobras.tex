\documentclass[12pt, a4paper]{article}
\usepackage[utf8]{inputenc}
\usepackage[T1]{fontenc}
\usepackage[brazil]{babel}
\usepackage{geometry}
\geometry{a4paper, left=3cm, right=2cm, top=3cm, bottom=2cm}
\usepackage{setspace}
\onehalfspacing
\usepackage{graphicx}
\usepackage{indentfirst}

\title{Relatório Técnico \\ Análise Organizacional da Petrobras}
\author{Giovanni Müller \\ RA: 1002351335}
\date{\today}

\begin{document}

\maketitle

\section*{Identificação}
\begin{tabular}{ll}
\textbf{Empresa:} & Petrobras (Petróleo Brasileiro S.A.) \\
\textbf{Setor Econômico:} & Petróleo, Gás e Energia \\
\end{tabular}

\section{Análise da Legalidade Jurídica: Contrato Social e Regime Societário}

De acordo com seu Estatuto Social, a Petrobras é uma \textbf{Sociedade Anônima (S/A)} de capital aberto. Esse regime societário é caracterizado pela divisão do capital em ações, que são negociadas no mercado de valores mobiliários. A empresa é, portanto, uma \textbf{empresa de grande porte}, constituída sob a forma de sociedade de economia mista, com controle majoritário exercido pela União Brasileira (Governo Federal). Suas ações são negociadas nas bolsas de valores do Brasil (B3), sob o ticker PETR3 e PETR4, e também no exterior (NYSE, New York Stock Exchange).

\section{Dimensões Legais}

A atuação da Petrobras é intensamente impactada por uma vasta gama de legislações específicas, dada a natureza crítica e estratégica de seu setor. As principais dimensões legais incluem:

\begin{itemize}
    \item \textbf{Ambiental:} A empresa está sujeita a uma rigorosa legislação ambiental, supervisionada pelo IBAMA (Instituto Brasileiro do Meio Ambiente e dos Recursos Naturais Renováveis) e ICMBio (Instituto Chico Mendes de Conservação da Biodiversidade). Suas operações, especialmente de exploração e produção de petróleo, devem seguir normas rígidas para prevenir e mitigar acidentes, como vazamentos de óleo.
    
    \item \textbf{Setor de Óleo e Gás:} É regulada pela \textbf{Agência Nacional do Petróleo, Gás Natural e Biocombustíveis (ANP)}, que fiscaliza todas as atividades do setor, desde a exploração até a comercialização de derivados, garantindo o cumprimento das políticas energéticas nacionais.
    
    \item \textbf{Trabalhista e Previdenciária:} Como uma das maiores empregadoras do país, deve seguir à risca a Consolidação das Leis do Trabalho (CLT) e as normas de segurança do trabalho, sendo também fiscalizada pelo Ministério do Trabalho e Emprego.
\end{itemize}

\section{Legalidade Técnica e de Certificação}

A Petrobras possui um robusto Sistema de Gestão Integrada, sendo certificada em diversas normas internacionais que atestam a qualidade e a segurança de seus processos. Dentre as certificações, destacam-se:

\begin{itemize}
    \item \textbf{ISO 9001} (Sistema de Gestão da Qualidade): Aplicada a diversos processos, unidades e refinarias, garantindo que produtos e serviços atendam aos requisitos de qualidade.
    
    \item \textbf{ISO 14001} (Sistema de Gestão Ambiental): Certifica que a empresa gerencia seus processos com foco na minimização de impactos ambientais.
    
    \item \textbf{ISO 45001} (Sistema de Gestão de Saúde e Segurança Ocupacional): Atesta o compromisso da empresa com a segurança e o bem-estar de seus colaboradores, contratados e comunidades do entorno.
    
    \item A empresa também possui outras certificações técnicas específicas do setor, como a \textbf{ISO 31000} (Gestão de Riscos).
\end{itemize}

\section{Legalidade de Validação Interna}

A Petrobras possui uma estrutura de governança corporativa bastante sólida e bem estabelecida, seguindo as melhores práticas do mercado. Sua estrutura é composta por:

\begin{itemize}
    \item \textbf{Conselho de Administração:} Órgão superior de deliberação, com membros eleitos pelos acionistas, incluindo representantes do controlador (União) e membros independentes.
    
    \item \textbf{Diretoria Executiva:} Responsável pela gestão dos negócios e pela implementação das estratégias aprovadas pelo Conselho.
    
    \item \textbf{Conselho Fiscal:} Atua na fiscalização dos atos dos administradores e na análise das demonstrações financeiras.
    
    \item \textbf{Comitês Especializados:} A empresa conta com comitês assessores ao Conselho de Administração, como o Comitê de Auditoria e Riscos, o Comitê de Pessoas e o Comitê de Estratégia, que aprofundam a análise e o controle sobre temas específicos.
\end{itemize}

Essa estrutura garante a validação interna dos processos, a gestão eficiente de riscos e a transparência perante acionistas, mercado e sociedade.

\end{document}
