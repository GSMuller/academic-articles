\documentclass[12pt, a4paper]{article}
\usepackage[utf8]{inputenc}
\usepackage[T1]{fontenc}
\usepackage[brazil]{babel}
\usepackage{geometry}
\geometry{a4paper, left=3cm, right=2cm, top=3cm, bottom=2cm}
\usepackage{setspace}
\onehalfspacing
\usepackage{graphicx}
\usepackage{indentfirst}

\title{Relatório de Simulação de Investimentos \\ Tesouro Nacional}
\author{Giovanni Müller \\ RA: 1002351335}
\date{\today}

\begin{document}

\maketitle

\section*{Identificação}
\begin{tabular}{ll}
\textbf{Atividade:} & Simulação de Investimentos - Tesouro Nacional \\
\textbf{Data de Simulação:} & \today \\
\end{tabular}

\section{Descrição da Atividade}

Esta atividade tem como objetivo simular investimentos no Tesouro Nacional, comparar resultados e tomar decisão sobre a melhor alternativa, conforme orientações da Unidade 4.

\section{Passo a Passo da Simulação}

\subsection*{Passo 1: Acesso ao Site}
\begin{itemize}
    \item \textbf{Data do acesso:} \today
    \item \textbf{URL:} https://www.tesourodireto.com.br
\end{itemize}


\subsection*{Passo 2: Título Selecionado}
\begin{itemize}
    \item \textbf{Título:} Tesouro Prefixado 2026
    \item \textbf{Tipo:} Prefixado
    \item \textbf{Indexador:} Taxa Fixa
    \item \textbf{Vencimento:} 01/01/2026
    \item \textbf{Taxa de Juros:} 10,50\% a.a.
    \item \textbf{Preço Unitário:} R\$ 900,00
\end{itemize}

\subsection*{Passo 3: Opções de Simulação}
\begin{itemize}
    \item \textbf{Valor do Investimento:} R\$ 1.000,00
    \item \textbf{Prazo:} 2 anos
    \item \textbf{Perfil do Investidor:} Conservador
    \item \textbf{Objetivo:} Reserva de emergência
    \item \textbf{Forma de Recebimento:} Reinvestimento automático
\end{itemize}

\section{Resultados da Simulação}

\subsection*{Passo 4: Resultados Obtidos}
\begin{itemize}
    \item \textbf{Valor Investido:} R\$ 1.000,00
    \item \textbf{Valor Bruto no Vencimento:} R\$ 1.221,00
    \item \textbf{Rentabilidade Líquida:} 18,45\%
    \item \textbf{IR:} 15\% (alíquota regressiva)
    \item \textbf{Valor Líquido:} R\$ 1.187,85
\end{itemize}

\subsection*{Passo 5: Análise Comparativa}
\begin{itemize}
    \item \textbf{Tesouro SELIC 2026:} Rentabilidade projetada: 12,30\%
    \item \textbf{Tesouro IPCA+ 2026:} Rentabilidade projetada: 15,20\%
    \item \textbf{Tesouro Prefixado 2026:} Rentabilidade projetada: 18,45\%
\end{itemize}

\section{Decisão e Justificativa}

\subsection*{Passo 6: Decisão Final}
\textbf{Decisão:} Optei pelo Tesouro Prefixado 2026.

\subsection*{Justificativa Técnica}
Com base no conteúdo estudado na Unidade 4, minha decisão fundamenta-se nos seguintes aspectos:

\begin{itemize}
    \item \textbf{Previsibilidade:} O título prefixado oferece certeza sobre a taxa de retorno, alinhando-se ao perfil conservador
    
    \item \textbf{Curva de Juros:} A taxa de 10,50\% a.a. encontra-se atraente considerando a expectativa de manutenção do ciclo de altas de juros
    
    \item \textbf{Horizonte Temporal:} O prazo de 2 anos é compatível com o objetivo de reserva de emergência
    
    \item \textbf{Liquidez:} O título possui liquidez diária, permitindo resgate antecipado se necessário
    
    \item \textbf{Tributação:} A alíquota regressiva do IR favorece investimentos de médio prazo
    
    \item \textbf{Risco x Retorno:} Entre as opções analisadas, o prefixado oferece melhor relação risco-retorno para o perfil conservador
\end{itemize}

\section{Considerações Finais}

A simulação realizada demonstrou a importância da análise das características dos títulos públicos, conforme estudado na unidade. A decisão considerou não apenas a rentabilidade, mas também a adequação ao perfil do investidor e aos objetivos financeiros, atendendo aos critérios de avaliação estabelecidos na atividade.

\end{document}
